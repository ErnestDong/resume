\documentclass{moderncv}
\moderncvstyle{banking}
\moderncvcolor{black}
\usepackage{xeCJK}
\usepackage[scale=0.8]{geometry}
\name{董}{晨阳}
\phone[mobile]{188~1159~9628}
\email{dongcy2018@econ.pku.edu.cn}
\photo{./董晨阳/证件照(白底).jpg}
\begin{document}
% \linespread{0.5}
\small{
    \maketitle
}
\section{教育背景}
\subsection{北京大学}
\cventry{2018 -- 2022 -- 2024}{风险管理与保险学}{经济学院}{硕士\&本科}{}{
    \cvitem{本科阶段}{GPA 3.73/4,排名 5/27。获得北京大学三等奖学金一次,校级三好学生一次}
    \cvitem{硕士阶段}{研一在读,GPA3.7/4,擅长程序设计}
}
\section{实习经历}
\subsection{广发基金}\cventry{2023/1 -- 至今}{研究发展部}{美股研究}{}{}{
    \begin{itemize}
        \item \textbf{《ChatGPT背后:LLM的发展与应用》}:大模型更需要数据、算力、人才和工程能力,AIGC早已到来
              \begin{itemize}
                  \item \textbf{MSFT vs GOOG}:AGI vs Pathways;ChatGPT源自技术积累组合,Google仍属领先但需时间训练
                  \item \textbf{NVDA \& MU}:数据和算力是AI产业卖铲人,CUDA是NVDA竞争壁垒,内存是AI芯片瓶颈
              \end{itemize}
        \item \textbf{《DevOps工具:现代软件开发的基础设施》}:DevOps是高质量软件开发的方式,上云是DevOps趋势
              \begin{itemize}
                  \item \textbf{TEAM}:高集成度与数据累积是其竞争优势;用户数自然增长、免费用户转化飞轮是其持续盈利的关键
                  \item \textbf{GTLB}:作为企业级安全和DevOps解决方案与GitHub差异化竞争,但竞争中营销费用沉重侵蚀利润
              \end{itemize}
        \item \textbf{其他研究工作}:PCAR;BILL,CFLT,FROG,PD等
    \end{itemize}
}

\subsection{中金公司}
\cventry{2022/7 -- 2022/11}{互联网组}{行业研究}{}{}{
    \begin{itemize}
        \item \textbf{《腾讯控股:一个观察框架》}:腾讯有望走出一个收入缓慢增长、利润较快增长、EPS 更快增长的范式
              \begin{itemize}
                  \item \textbf{理解框架}:腾讯用社交网络形成人与人的连接,商业模式成功取决于与社交网络的配合
                  \item \textbf{底层基础}:短视频并未真正动摇腾讯社交网络,尽管有分流,腾讯以全新的形态拥抱了短视频的趋势
                  \item \textbf{利润改善}:增量业务对利润的弹性更大,降本增效强化了增长模型,相比收入,利润改善趋势更为确定
                  \item \textbf{潜在催化}:稳定现金流,以及规模可观的投资组合,回购或分红助力股东价值得以更好释放
              \end{itemize}
        \item \textbf{《Web3.0社区底座,以太坊升级在即》}:公链承担基础设施职能,预计形成一超(以太坊)多强格局
        \item \textbf{其他研究工作}:《腾讯控股3Q22业绩点评》;《中金看海外:BeReal\&Gas》;《线上内容平台回顾-2Q22》等
    \end{itemize}
}
\subsection{淡水泉投资}
\cventry{2021/12 -- 2022/06}{投资部}{数据分析}{}{}{
    \begin{itemize}
        \item 编写回测系统与ETF统计系统,研究A股风格切换;分解基金风格因子,估计基金因子暴露度
        \item 利用另类数据拟合不同品牌不同车型的新能源车销量与上险量,对新势力预测效果较好
    \end{itemize}
}
% \subsection{嘉实基金}
% \cventry{2021/07 -- 2021/11}{固收研究部}{固收量化}{}{}{
%     \begin{itemize}
%         \item 负责量化评级模型升级迭代,协助机器学习评级模型的开发以及部分算法设计,提升程序运行性能
%         \item 对债券成交价与估值的偏离进行回归分析,设计信用债报价系统,并研究信用债违约特征
%     \end{itemize}
% }
\subsection{易方达基金}
\cventry{2020/09 -- 2021/01}{混合资产部}{行业研究}{}{}{
    \begin{itemize}
        \item 研究全球半导体周期趋势,定量分析我国半导体行业产出增加对GDP及工业增加值的影响
        \item 追踪新冠疫苗进展,爬取疫情动态,分析预测美国新冠疫情爆发概率
    \end{itemize}
}
\section{课外活动与技能}
\cvitem{2021“美国风险与保险协会(RIMS)”年度峰会案例大赛}{\small 作为唯一美国本土外队伍进入全球前八强}
\cvitem{“一桌”支教计划}{\small 2020年疫情爆发初期作为第一批支教老师,辅导武汉医护高三子女课业}
% \cvitem{怀源计划暑期实践}{\small 前往甘肃省木耳乡支教一个月}
\cvitem{技能}{\small Python、Wind、Office、C/C++、R、SQL、Solidity、JavaScript、Bloomberg、Tableau等}
\end{document}
