\documentclass[a4paper]{moderncv}
\moderncvstyle{banking}
\moderncvcolor{black}
\usepackage{xeCJK}
\usepackage[scale=0.9]{geometry}
\geometry{a4paper,top=0cm,bottom=1cm}
\name{董}{晨阳}
\phone[mobile]{188~1159~9628}
\email{dongcy2018@econ.pku.edu.cn}

\geometry{a4paper,top=0cm,bottom=1cm}
\photo{./董晨阳/证件照(白底).jpg}
\patchcmd{\makehead}
{\hfil}
{\hspace*{0.15\textwidth}}
{}
{}
\patchcmd{\makehead}
{\setlength{\makeheaddetailswidth}{0.8\textwidth}}
{\setlength{\makeheaddetailswidth}{0.67\textwidth}}
{}
{}
\patchcmd{\makehead}
{\\[2.5em]}
{\hfil\raisebox{-.7cm}{\includegraphics[width=\@photowidth]{\@photo}}\\[2.5em]}
{}
{}
\begin{document}
\small{
    \maketitle
}
\section{教育背景}
\subsection{北京大学}
\cventry{2022 -- 2024}{风险管理与保险学}{经济学院}{硕士}{}{
    \cvitem{GPA}{GPA 3.7/4,排名前 20\%}
    \cvitem{奖励荣誉}{获得北京大学三等奖学金,校级三好学生}
}
\subsection{北京大学}
\cventry{2018 -- 2022}{风险管理与保险学}{经济学院}{本科}{}{
    \cvitem{GPA}{GPA 3.73/4,排名前 20\%,保研至本院}
}
\section{实习经历}
\subsection{广发基金}
\cventry{2023/7 -- 2023/8}{研究发展部}{行业研究}{}{}{
    \begin{itemize}
        \item \textbf{Elastic N.V.}:AI应用爆发将推动Elastic困境反转,预期FY24较一致预期双miss而FY25双beat,建议持有
              \begin{itemize}
                  \item \textbf{向量数据库:强势行业的潜在领导者}。与AI应用共荣,TAM 64亿美元。预期凭借稳定性和服务能力市场份额10\%-15\%
                  \item \textbf{可观测性:弱势行业的幸存者}。因云上支出放缓而承压,且未来复苏程度有限,但开源卡位+AI催化不会出清
                  \item \textbf{潜在催化}:Serverless将带动开源用户转化;AI Assistant与OTel标准弥补分析短板,凭借开源、低价获取份额
              \end{itemize}
    \end{itemize}
}
\subsection{中欧基金}
\cventry{2023/5 -- 2023/6}{研究部}{行业研究}{}{}{
    \begin{itemize}
        \item \textbf{深信服}:网安企业竞争从拼产品力到拼渠道能力,战略层面对渠道的管控力度边际加强、竞争环境边际改善,建议持有
              \begin{itemize}
                  \item \textbf{短期基本盘复苏确定性高}:竞争环境边际改善,且公司的狼性基因决定其能凭借加大对渠道的管控力度来解决问题
                  \item \textbf{中长期,安全SaaS有望重塑估值体系}:深信服具备客户卡位优势,当期核心矛盾看核心腰部客群渗透
              \end{itemize}
    \end{itemize}
}
\subsection{广发基金}
\cventry{2023/1 -- 2023/4}{研究发展部}{行业研究}{}{}{
    \begin{itemize}
        \item \textbf{AI专题系列报告}:大模型最重要的是数据和算力、其次是人才和工程能力及模型,AIGC时代在ChatGPT前早已到来
              \begin{itemize}
                  \item \textbf{模型层}:ChatGPT是大模型商业化而非AI模型的奇点时刻,Google的技术仍然领先只是数据与微调的时间不足
                  \item \textbf{硬件层}:Nvidia依靠软硬一体的优化、十年投入的生态以及开发全流程的覆盖,塑造了其在GPGPU领域的护城河
                  \item \textbf{软件层}:调用API、做提示工程较难成为竞争优势;AI框架逐渐大一统,AI开发工具软件将迎来黄金时代
                        %   \item \textbf{国内机会}:大模型仍有代差;ASIC大有可为,在AI芯片创业公司方兴未艾背景下,看好国产IP与Chiplet厂商
              \end{itemize}
        \item \textbf{《TEAM:从敏捷开发、DevOps到MLOps》}:尽管受宏观需求疲软影响,公司基本面依然坚实,免费用户转化可持续
              \begin{itemize}
                  \item \textbf{不同于市场的观点}:转化飞轮受宏观及科技公司裁员影响,项目管理是软件开发刚需,付费用户将依旧可持续自然增长
                  \item \textbf{竞争优势}:高集成度的Jira成为DevOps工作流核心,数据积累提升用户粘性、免费用户转化飞轮大幅降低营销费用
                  \item \textbf{未来增长}:MLOps管理ML应用生命周期,本质还是基于DevOps,可能成为公司在AI时代的新增长曲线
              \end{itemize}
        \item 其他研究工作:重卡行业PCAR;SaaS行业BILL,CFLT,FROG,PD,AI,WIX,GDDY等
    \end{itemize}
}

\subsection{中金公司}
\cventry{2022/7 -- 2022/11}{互联网组}{行业研究}{}{}{
    \begin{itemize}
        \item \textbf{《腾讯控股:一个观察框架》}:腾讯有望走出一个收入缓慢增长、利润较快增长、EPS 更快增长的范式,更像一个价值股
              %   \begin{itemize}
              %       \item \textbf{理解框架}:腾讯利用社交网络形成人与人的连接,游戏、广告等商业模式成功取决于与社交网络流量的配合
              %       \item \textbf{底层基础}:短视频并未真正动摇腾讯社交网络,尽管有一定的分流,腾讯以全新的形态拥抱了短视频的趋势
              %       \item \textbf{利润改善}:相比收入,利润改善趋势更为确定,降本增效强化了增长模型,视频号增量业务对利润的弹性更大
              %       \item \textbf{潜在催化}:尽管大股东持续减持,稳定的现金流、规模可观的投资组合以及回购、分红助力股东价值更好释放
              %   \end{itemize}
        \item \textbf{《Web3.0社区底座,以太坊升级在即》}:公链承担基础设施职能,预计形成一超(以太坊)多强格局
        \item 其他研究工作:《腾讯控股3Q22业绩点评》;《中金看海外:BeReal\&Gas》;《线上内容平台回顾-2Q22》等
    \end{itemize}
}
% \subsection{淡水泉投资}
% \cventry{2022/03 -- 2022/06}{数据组}{数据分析}{}{}{
%     \begin{itemize}
%         \item 新能源车销量预测:结合另类数据与高频数据,分析不同品牌、不同车型的新能源车的销量与上限量
%         \item A股风格切换:根据A股近10年成交数据,编写回测系统与ETF统计系统,研究价值-成长风格切换
%     \end{itemize}
% }
% \subsection{嘉实基金}
% \cventry{2021/07 -- 2021/11}{固收研究部}{固收量化}{}{}{
%     \begin{itemize}
%         \item 负责量化评级模型升级迭代,协助机器学习评级模型的开发以及部分算法设计,提升程序运行性能
%         \item 对债券成交价与估值的偏离进行回归分析,设计信用债报价系统,并研究信用债违约特征
%     \end{itemize}
% }
% \subsection{易方达基金}
% \cventry{2020/09 -- 2021/01}{混合资产部}{行业研究}{}{}{
%     \begin{itemize}
%         \item 半导体周期:全球需求、供给和库存周期共振,且历史上半导体周期启动均有契机;我国电子行业对经济产出定量分析
%         \item 全球疫情追踪:追踪全球新冠疫苗进展,爬取疫情动态,分析预测美国新冠疫情爆发概率
%     \end{itemize}
% }
\section{课外活动与技能}
\cvitem{2021“美国风险与保险协会(RIMS)”年度峰会案例大赛}{\small 作为唯一美国本土外队伍进入全球前八强}
\cvitem{“一桌”支教计划}{\small 2020年疫情爆发初期作为第一批支教老师,线上辅导武汉医护高三子女课业}
\cvitem{怀源计划暑期实践}{\small 前往甘肃省木耳乡小学支教考察两个月}
\cvitem{技能}{\small Python、Wind、Office、C/C++、R、SQL、Solidity、JavaScript、Bloomberg、Tableau等}
\end{document}
